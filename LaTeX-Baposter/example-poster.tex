\documentclass[
  a0paper,
  portrait,
  fontscale=.35 % scales inversely (larger value=smaller font). Default is 0.292. Also affects Title etc!
  ]{baposterrptu} 

\usepackage{graphicx}             % Include Graphics
\usepackage{graphbox}	            % Alignment options for graphics
\usepackage{rptu-poster}    % some commands and color definitions

%choose your poster colors here:
\colorlet{color_background}{apfel}
\colorlet{color_accents}{petrol}
% all university colors are available:
% schiefer, ozean ,nacht ,tag, petrol, pflaume, fuchsia, himbeere, mango
% ToDo: Automize pair color selection. 

\begin{document}
\begin{poster}{
    % here you can add baposter arguments to change the layout e.g. columns=3
  }
  %
  % the template does not support a logo in the top left (don't try, it's disabled!)
  {}
  %
  % the poster title
  {Some Title that is very cool }
  %
  % the author(s)
  {Some Author}
  %
  % RPTU Logo top right. Replace with your own logo if required. Note that only RPTU-like Logos are permitted here.
  {\rptuLogo}
  %
  % The footer. Place logos of affiliations etc here.
  \footer
  {\hfill
    \includegraphics[align=c,scale=.3]{example-image-b}
    \hfill
    \color{white} \huge www.github.com/Patschke/RPTU-Design
    \hfill
    \includegraphics[align=c,scale=.3]{example-image-c}
    \hfill}
  %
  % Your content starts from here. 
  %
  \begin{posterbox}[name=intro,column=0,row=0]{Introduction}

    This is a poster box

    \begin{itemize}
      \item Use \LaTeX code as usual
      \item Write stuff
      \item Give me feedback if you don't like this.
      \item You can also let me know if you do like it.
            \begin{itemize}
              \item As usual, you can stack orders of itemize.
            \end{itemize}
    \end{itemize}

  \end{posterbox}

  \begin{posterbox}[name=test,column=1,row=0, bottomaligned=intro]{A Box}

    Another poster box. Bottom-aligned with the introduction box. To show how nicely \LaTeX renders math, here is some equation:
    $$\int_{-\infty}^{\infty}e^{-x^2}\sin{\frac{1}{x}}dx =da \alpha\beta\gamma\Psi\Gamma$$

  \end{posterbox}

  \begin{posterbox}[name=test2,column=0,below=intro, span=2]{A wide Box}

    A third poster box

    Use the column, row, span, above, below and bottomaligned keys to position boxes.

  \end{posterbox}

  \begin{posterbox}[above=foot, below=test2, column=1]{Outlook}
    And yet another one, stretching down to the foot.

    I could add some useful how-to-use guide here, but than this would be more like an options.tex and less an example. So meh.
  \end{posterbox}

\end{poster}
\end{document}
