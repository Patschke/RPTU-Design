\documentclass[
  a0paper,
  portrait,
  fontscale=.35 % scales inversely (larger value=smaller font). Default is 0.292. Also affects Title etc!
  ]{baposterrptu} 

\usepackage{graphicx}             % Include Graphics
\usepackage{graphbox}	            % Alignment options for graphics
\usepackage{rptu-poster}          % some commands and color definitions
\usepackage{qrcode}               % QR Codes
\usepackage{lipsum}                % Dummy text

%choose your poster colors here:
\colorlet{color_background}{apfel}
\colorlet{color_accents}{petrol}
% all university colors are available:
% schiefer, ozean ,nacht ,tag, petrol, pflaume, fuchsia, himbeere, mango
% ToDo: Automize pair color selection. 

\begin{document}
\begin{poster}{
    % here you can add baposter arguments to change the layout e.g. columns=3
    % this example just uses all the defaults, so no need to specify anything
    % as I know that people will complain about not enough space and try to
    % pack stuff clother, here are some example options for that
    % outercolspacing=2.5em, % adjust the white space on the left/right edges
    % boxpadding=1.5em,      % adjust the spacing within a box
    % boxheaderheight=3.5em, % change the header size
  }
  %
  {}% There used to be an top-left logo, but the current corporate design doesn't have that. 
  %
  % the poster title/Subtitle. 
  {Some Title that is very interesting\\
    \sf Or at least funny?}
  %
  % the author(s)
  {Patrick Mischke}
  %
  % RPTU Logo top right. Replace with your own logo if required. Note that only RPTU-like Logos are permitted here.
  {\rptuLogo}
  %
  % Footer bottom left. This is supposed to hold contact details, but you can of course put anything here.
  {
    \begin{minipage}{.8\footerheight}
      \qrcode[height=.8\footerheight]{https://github.com/Patschke/RPTU-Design}
    \end{minipage}
    \hfill
    \begin{minipage}{.4\paperwidth}
      \textbf{License:}\\
      Stuff like the logos are probably owned by the RPTU.\\\\
      \textbf{Contact:}\\
      Please open an issue at GitHub or email me.
    \end{minipage}
  }
  % Footer bottom right. This is supposed to hold references, but you can of course put anything here.
  {
    References:
    \begin{itemize}
      \item https://github.com/Patschke/RPTU-Design
      \item https://rptu.de/intern/brand-portal
      \item https://github.com/RedHatOfficial/RedHatFont
      \item Original baposter by Brian Amberg
    \end{itemize}
  }
  %
  % Your content starts from here. 
  %
  \begin{posterbox}[name=intro,column=0,row=0]{Introduction}

    This is a poster box. Here are some features and examples on how to use this template:

    \begin{itemize}
      \item Use \LaTeX-code as usual
      \item Write stuff
      \item Give me feedback if you don't like this.
      \item You can also let me know if you do like it.
            \begin{itemize}
              \item As usual, you can stack orders of itemize.
            \end{itemize}
    \end{itemize}

  \end{posterbox}

  \begin{posterbox}[name=box,column=1,row=0, bottomaligned=intro]{A Box}

    Another poster box. Use the column, row, span, above and below keys to position boxes. There are certainly tutorials out there how to use baposter, and I tried not to break the functionality it has.


    To show how nicely \LaTeX renders math, here is some equation:
    $$\int_{-\infty}^{\infty}e^{-x^2}\sin{\frac{\alpha}{x}}dx = \beta$$

    You may have noticed, that not all the above is rendered in RedHatText. This is due to a lack of Greek letters, so RedHat is just used for latin characters and numbers.
  \end{posterbox}

  \begin{posterbox}[name=widebox,column=0,below=intro, span=2]{A wide Box}

    A third poster box. It's wider than the other ones, as `span=2' is set. This allows you to write even more complicated math!

    \begin{align*}
      \int_{0}^{1} \left( \sum_{n=1}^{\infty} \frac{(-1)^{n+1}}{n} x^n \right) dx + \sum_{k=1}^{m} \frac{1}{k^2} \left( \int_{a}^{b} e^{kx} \, dx \right) + \prod_{i=1}^{n} \left( 1 + \frac{1}{i^2} \right)
      + \lim_{x \to \infty} \left( \frac{\sqrt{x^2 + 1}}{x + 1} \right) \cdot \int_{-\infty}^{\infty} \left( \frac{1}{\sqrt{2\pi\sigma^2}} e^{-\frac{(t-\mu)^2}{2\sigma^2}} \right) dt = \sum_{j=1}^{\infty} \left( \frac{(-1)^j}{j^2} \right) \\
      \quad \stackrel{!}{=} \iint_{R} \left( x^2 + y^2 \right) \, dA - \frac{d}{dx} \left( x^x \right) + \oint_{\gamma} \frac{1}{z} \, dz + \sum_{n=0}^{\infty} \frac{(-1)^n}{(2n+1)!} x^{2n+1} - \int_{-\infty}^{\infty}e^{-x^2}\sin{\frac{\alpha}{x}}dx
    \end{align*}

    You have to love that, right?
  \end{posterbox}

  \begin{posterbox}[below=widebox, column=0]{Images}
    Of course, common \LaTeX-packages for e.g. images work as usual here. Sou you can just copy and paste the stuff directly from your last paper. Awesome!\vspace{1em}

    \includegraphics[width=\textwidth]{example-image}\vspace{1em}

    I don't know what else to write, so here is some lipsum:
    \lipsum[1][1-5]
  \end{posterbox}

  \begin{posterbox}[name=outlook, below=widebox, column=1]{Outlook}
    Well, I guess we could have... More cool posters with this template? That would be nice.\vspace{1em}

    Again, I don't like white space, so here is some filler:
    \lipsum[1]
  \end{posterbox}

  \begin{posterbox}[above=bottom, column=1]{Other Logos}
    There is no specific location within the poster for these logos elsewhere (e.g. in the footer). So just place them here at the bottom right.\vspace{1em}

    \includegraphics[width=.3\textwidth]{example-image-a}\hfill
    \includegraphics[width=.3\textwidth]{example-image-b}\hfill
    \includegraphics[width=.3\textwidth]{example-image-c}
  \end{posterbox}

\end{poster}
\end{document}