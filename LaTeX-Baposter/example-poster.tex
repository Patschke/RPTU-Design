\documentclass[
  a0paper,
  portrait,
  fontscale=.35 % scales inversely (larger value=smaller font). Default is 0.292. Also affects Title etc!
  ]{baposterrptu} 

\usepackage{graphicx}             % Include Graphics
\usepackage{graphbox}	            % Alignment options for graphics
\usepackage{rptu-poster}          % some commands and color definitions
\usepackage{qrcode}               % QR Codes

%choose your poster colors here:
\colorlet{color_background}{apfel}
\colorlet{color_accents}{petrol}
% all university colors are available:
% schiefer, ozean ,nacht ,tag, petrol, pflaume, fuchsia, himbeere, mango
% ToDo: Automize pair color selection. 

\begin{document}
\begin{poster}{
    % here you can add baposter arguments to change the layout e.g. columns=3
    % this example just uses all the defaults, so no need to specify anything
  }
  %
  {}% There used to be an top-left logo, but the current corporate design doesn't have that. 
  %
  % the poster title
  {Some Title that is very cool}
  %
  % the author(s)
  {Patrick Mischke}
  %
  % RPTU Logo top right. Replace with your own logo if required. Note that only RPTU-like Logos are permitted here.
  {\rptuLogo}
  %
  % Footer bottom left. This is supposed to hold contact details, but you can of course put anything here.
  {
    \begin{minipage}{.8\footerheight}
      \qrcode[height=.8\footerheight]{https://github.com/Patschke/RPTU-Design}
    \end{minipage}
    \hfill
    \begin{minipage}{.4\paperwidth}
      \textbf{License:}\\
      Stuff like the logos are probably owned by the RPTU.\\\\
      \textbf{Contact:}\\
      Please open an issue at GitHub or email me.
    \end{minipage}
  }
  % Footer bottom right. This is supposed to hold references, but you can of course put anything here.
  {
    References:
    \begin{itemize}
      \item https://github.com/Patschke/RPTU-Design
      \item https://rptu.de/intern/brand-portal
      \item https://github.com/RedHatOfficial/RedHatFont
      \item Original baposter by Brian Amberg
    \end{itemize}
  }
  %
  % Your content starts from here. 
  %
  \begin{posterbox}[name=intro,column=0,row=0]{Introduction}

    This is a poster box

    \begin{itemize}
      \item Use \LaTeX code as usual
      \item Write stuff
      \item Give me feedback if you don't like this.
      \item You can also let me know if you do like it.
            \begin{itemize}
              \item As usual, you can stack orders of itemize.
            \end{itemize}
    \end{itemize}

  \end{posterbox}

  \begin{posterbox}[name=test,column=1,row=0, bottomaligned=intro]{A Box}

    Another poster box. Bottom-aligned with the introduction box. To show how nicely \LaTeX renders math, here is some equation:
    $$\int_{-\infty}^{\infty}e^{-x^2}\sin{\frac{1}{x}}dx =da \alpha\beta\gamma\Psi\Gamma$$

  \end{posterbox}

  \begin{posterbox}[name=test2,column=0,below=intro, span=2]{A wide Box}

    A third poster box

    Use the column, row, span, above, below and bottomaligned keys to position boxes.

  \end{posterbox}

  \begin{posterbox}[below=test2, column=1]{Outlook}
    And yet another one. Awesome, right?

    I could add some useful how-to-use guide here, but than this would be more like an options.tex and less an example. So meh.
  \end{posterbox}

\end{poster}
\end{document}