% !TeX program = LuaLaTeX
%
% THE BEER-WARE LICENSE (Rev. 42):
% Ronny Bergmann <bergmann@mathematik.uni-kl.de> wrote this file. As long as you
% retain this notice you can do whatever you want with this stuff. If we meet
% some day, and you think this stuff is worth it, you can buy me a beer or
% coffee in return.
%
% 
%
% Actually Ronny Bergmann wrote a template file for the TUK. Patrick Mischke 
% <mischke@rptu.de> rewrote it for the RPTU. I don't drink beer or coffee, 
% but I'm sure you can buy me something nice as well if you feel like it. 
%
% This file is just to get started - You need the corresponding Logo
%
\documentclass[german,10pt,xcolor=colortbl,compress,aspectratio=169
%,draft
]{beamer}
\usepackage[utf8]{inputenc}
\usepackage[OT1]{fontenc}
\usepackage{calc}
\usepackage[ngerman]{babel} % Neue Rechtschreibung
\usepackage{amsmath,amsthm,amssymb,euscript} % AMS-LaTeX  
\usepackage{enumerate,graphicx}
% Load Theme
\usetheme[
	% The Corporate Design does not use a Navigation bar, but if you like it here you go
	navigation=false,
	frametotal=true
    ,maincolor=rptupink
	%,twoColorBoxes
	,oneColorBoxes
	,signetontitle=true
]{RPTU}
%
\title{Beispielpräsentation}
\subtitle{Untertitel}
\date[]{\today\\[1ex] WorkShop/Conference}
\author[Autor in Fußzeile]{Autor: Name, Vorname}
\institute[]{AG xy\\FB ab\\RPTU in Kaiserslautern}
%Setze ein Logo oben rechts
%\logo{\includegraphics[width=2cm]{example-image-a}}

\begin{document}
\maketitle
\section{Einleitung}
\begin{frame}{Inhalt}
	\tableofcontents
\end{frame}
\begin{frame}{Es geht los}
	Hier wozu es dieses Template gibt:
	\begin{itemize}
		\item Präsentationen bauen
		      \begin{itemize}
			      \item In \LaTeX
			      \item Und trotzdem im CD
		      \end{itemize}
		\item Gut aussehende Vorträge.
	\end{itemize}
	So benutzt du es:
	\begin{enumerate}
		\item Git Repository klonen
		\item \LaTeX-Präsentation bauen
		\item Genialen Vortrag halten
	\end{enumerate}
\end{frame}
\section{Erstes Thema}
\begin{frame}{Ein erstes Thema}{Mit Untertitel}
	\begin{lemma}[Ein Beispiellemma]
		Ist das hier und es gilt. Vielleicht.
	\end{lemma}
	\begin{example}
		Ein Beispiel, wie dieses hier
	\end{example}
	\alert{ACHTUNG!}
	Etwas Hervorgehobenes. Die Farben auf den Folien sind ein Farb-Paar aus dem Corporate Design.\\

\end{frame}
\begin{frame}{Ein Beweis}{Mit Untertitel}
	\begin{proof}
		Weil.
	\end{proof}
\end{frame}

\begin{frame}[plain]{}{}%Ohne Titel und Untertitel damits ganz leer und weiß ist
	Ich bin so ein leerer Frame
\end{frame}

\begin{frame}[coloredbackground]{Bunt}
	Dies ist ein Frame mit buntem Hintergrund.
\end{frame}
\begin{frame}{Abschluss}
	Danke für die Aufmerksamkeit und so!
\end{frame}
\end{document}