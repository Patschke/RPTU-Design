%
% THE BEER-WARE LICENSE (Rev. 42):
% Ronny Bergmann <bergmann@mathematik.uni-kl.de> wrote this file. As long as
% you retain this notice you can do whatever you want with this stuff. If we
% meet some day, and you think this stuff is worth it, you can buy me a beer
% or coffee in return.
%
%
% Actually Ronny Bergmann wrote a template file for the TUK. Patrick Mischke 
% <mischke@rptu.de> rewrote it for the RPTU. I don't drink beer or coffee, 
% but I'm sure you can buy me something nice as well if you feel like it. 
%
% This file is just to get started - You need the corresponding Logo
%
\documentclass[german,10pt,xcolor=colortbl,compress
%,draft
]{beamer}
\usepackage[OT1]{fontenc}
\usepackage{calc}
\usepackage[ngerman]{babel} % Neue Rechtschreibung
\usepackage{amsmath,amsthm,amssymb,euscript} % AMS-LaTeX  
\usepackage{enumerate,graphicx,booktabs}
\usepackage[TeX]{listings}
\lstset{basicstyle=\small\ttfamily}
\RequirePackage[no-math]{fontspec} 
\RequirePackage{xltxtra}
\defaultfontfeatures{Mapping=tex-text} % converts LaTeX specials (``quotes'' --- dashes etc.) to unicode
% Load Them
\usetheme[displaynavigation,reducedframetotal,maincolor=rptupink
]{RPTU}
% Meta alternative in XeLaTeX
\setbeamertemplate{navigation symbols}{}
\title{\LaTeX-Beamer Theme im Stil der \mbox{RPTU Kaiserslautern-Landau}}
\subtitle{Verwendung und Optionen dieses Themes}
\date[]{\today}
\author[P. Mischke]{Patrick Mischke\\Original für die TUK von Ronny Bergmann}
\institute[]{AG Ott\\FB Physik\\RPTU Kaiserslautern}
%\hypersetup{colorlinks=true, linkcolor=tuklblue, urlcolor=blue!50!black}

\renewcommand{\theSecondLogo}{\includegraphics[width=.15\paperwidth]{logos/Logo_FKZM_rgb.jpg}}

\begin{document}
\begin{frame}{Versionshinweis}
	Dieses Design ist noch Work-In-Progress. Es sollte aber schon brauchbar nutzbar sein, einige Nice-To-Have Feature folgen hoffentlich noch. \\
	Wenn du anregungen und Feedback hast, melde dich gerne. Es lohnt sich womöglich, ab und an die neuste Version herunterzuladen.
\end{frame}
\maketitle
\begin{frame}{Inhaltsverzeichnis}
	\tableofcontents
\end{frame}
\section{Einrichten}
\begin{frame}[fragile]{Benötigte Pakete}
	folgende Pakete werden von dem Theme benötigt und per \lstinline!\RequirePackage! eingebunden:
	\begin{itemize}
		\item \lstinline!ifthen!
		\item \lstinline!pdftexcmds!
		\item \lstinline!calc!
		\item \lstinline!TikZ!
	\end{itemize}\vspace{\baselineskip}
	sowie bei LuaLaTeX bzw. XeLaTeX zusätzlich
	\begin{itemize}
		\item \lstinline!fontspec!
	\end{itemize}\vspace{\baselineskip}
	Wird LuaLaTeX bzw. XeLaTeX verwendet, wird die Uni Schriftart RedHatText\footnote{\url{https://github.com/RedHatOfficial/RedHatFont/releases/}} verwendet.
	Diese muss dafür im System installiert sein.
\end{frame}
\begin{frame}{Einbinden des Themes}
	\begin{itemize}
		\item Benötigt: 4 Dateien
		      \begin{itemize}
			      \item \lstinline|beamercolorthemeRPTU.sty|
			      \item \lstinline|beamerinnerthemeRPTU.sty|
			      \item \lstinline!beamerouterthemeRPTU.sty!
			      \item \lstinline|beamerthemeRPTU.sty|
		      \end{itemize}

		\item nach den Paketen mit \textbackslash\lstinline|usetheme[<Optionen>]\{RPTU\}| das Paket laden
		\item Am einfachsten: Mit dem Beispiel \lstinline!example-pres-RPTU-design.tex! loslegen
	\end{itemize}
\end{frame}
\section{Optionen}
\subsection*{}
\newcommand{\farbe}[1]{{\color{#1}\lstinline|#1|}}
\begin{frame}{Optionen}{Farben}
	Die Präsentation wird mit einem der Farbpaare aus dem Corporate Design erstellt. Über die Option \lstinline!maincolor=<farbe>! kann die Hauptfarbe gewählt werden. Wird die Option nicht verwendet, wird die Farbe zufällig ausgesucht. \par
	Folgenden Farben sind möglich:
	\farbe{rptublaugrau} \farbe{rptugruengrau} \farbe{rptudunkelblau} \farbe{rptuhellblau}
	\farbe{rptudunkelgruen} \farbe{rptuhellgruen} \farbe{rptuviolett} \farbe{rptupink}
	\farbe{rpturot} \farbe{rptuorange}. Experimentell sind auch \farbe{rptuschwarz} und \color{black}\lstinline!rptuweiss! nutzbar.\par
	Die passende Zweitfarbe des Paares wird automatisch gesetzt. Innerhalb der Präsentation kann über obige Namen auf die Farben zugegriffen werden. Die gewählten Farben stehen per \farbe{mainrptuthemecolor} und \farbe{secondaryrptuthemecolor} zur Verfügung. \par
\end{frame}
\begin{frame}{Schriftart}{nur XeLaTeX/LuaLaTex}
	\begin{itemize}
		\item Wird XeLaTeX oder LuaLaTex verwendet, dann wird automatisch RedHatText als Schriftart ausgewählt.
		\item Sollte RedHatText nicht installiert sein, führt dies zu Fehlern.
		\item Wird \lstinline!pdflatex! verwendet, so kann RedHatText nicht verwendet werden.
	\end{itemize}
\end{frame}
\begin{frame}{Gesamtfolienanzahl und Navigation}
	\begin{itemize}
		\item \lstinline|frametotal=true|: Zeige hinter der Foliennummer in der Fußzeile die Gesamtfolienanzahl. Beispiel: \insertframenumber{}/\inserttotalframenumber
		\item \lstinline|frametotal=false|: Zeige die Gesamtanzahl nicht. Beispiel: \insertframenumber{}
		\item Standardwert: \lstinline|frametotal=true|
		\item Bonusoption: \lstinline|reducedframetotal|. Zeige die gekürzte Folienzahl. Beisipel siehe diese Folien ;-)
	\end{itemize}
	\vspace{\baselineskip}
	\begin{itemize}
		\item \lstinline|navigation=true| bzw. \lstinline|displaynavigation| Zeige in der Kopfzeile eine Navigation an (wie in diesem Foliensatz)
		\item \lstinline|navigation=false| bzw. \lstinline|hidenavigation| Zeige keine Navigation an
		\item Standardwert: \lstinline|navigation=true|
	\end{itemize}
\end{frame}
\begin{frame}[plain]{Leere Folien}
	Wird einmal mehr Platz benötigt, kann mit der option \lstinline|[plain]| des \lstinline|frame| eine komplett leere Folie erzeugt werden, wie diese hier.
\end{frame}
\begin{frame}{Darstellung von Blöcken}
	\begin{itemize}
		\item \lstinline!twoColorBoxes=true! (default): Boxen werden zweigeteilt dargestellt.
		      \begin{block}{Beispiel}
			      Dies ist eine zweifarbige Box.
		      \end{block}
		\item \lstinline!twoColorBoxes=false! oder \lstinline!oneColorBoxes!: Boxen werden einfarbig dargestellt.
		      \setbeamercolor*{box text}{fg=white}
		      \setbeamercolor*{block body}{parent=box text, use=block title,bg=block title.bg}
		      \setbeamercolor*{block body alerted}{parent=box text,use=block title alerted,bg=block title alerted.bg}
		      \setbeamercolor*{block body example}{parent=box text,use=block title example,bg=block title example.bg}
		      \begin{block}{Beispiel}
			      Dies ist eine einfarbige Box.
		      \end{block}
	\end{itemize}
\end{frame}
\begin{frame}{Seitenverhältnis}
	\begin{itemize}
		\item Diese Vorlage sollte 16:9 und 4:3 Formate unterstützen.
		\item Auswahl direkt in den Optionen von \lstinline!beamer! in der \lstinline!documentclass[...]!, also z.B. \lstinline!aspectratio=169!
		\item Man kann auch beliebige andere Seitenverhältnisse setzen (also z.b. \lstinline!aspectratio=137! für 13:7 oder \lstinline!aspectratio=42! für 4:2 oder \lstinline!aspectratio=1024! für 10:24). Wie gut das Design dann funktioniert mag ich nicht vorhersagen.
		\item Standardwert von Beamer ist 4:3
	\end{itemize}
\end{frame}
\section{Befehle}
\subsection*{}
\begin{frame}[fragile]{Titelseite}
	Der Befehl \lstinline|\maketitle| erzeugt eine Titelfolie.

	Er kann in einem \lstinline|frame| aufgerufen werden, dann wird der Stil des Frames verwendet.

	Außerhalb eines Frames wird ein Titelframe erzeugt, der den Standardkopf hat, aber keine Fußzeile (siehe Titelfolie dieser Folien).
\end{frame}
\begin{frame}[fragile]{Ein zweites Logo auf der Titelseite}
	Dies funktioniert leider noch nicht. Ich implementiere es aber bestimmt noch irgendwann.

	Der Befehl \lstinline|\theSecondLogo| setzt ein neues Logo, indem man diesen umdefiniert.

	Dabei wird in dem Befehl die gesamte Grafik eingebunden; in diesen Folien etwa das Logo des Felix-Klein-Zentrums mittels

	\begin{lstlisting}
\renewcommand{\theSecondLogo}%
	{\includegraphics[width=.15\paperwidth]{logos/Logo_FKZM.jpg}}
		\end{lstlisting}
\end{frame}
\begin{frame}{Links}
	\begin{itemize}
		\item Vorlagen und Arbeitshilfen im Design der RPTU\\ \href{https://www.startklar2023.de/brand-portal-rptu/}{https://www.startklar2023.de/brand-portal-rptu/}
		\item Die Schriftart RedHatText\\
		      \href{https://github.com/RedHatOfficial/RedHatFont/}{https://github.com/RedHatOfficial/RedHatFont/}
		\item Userguide zu \LaTeX-Beamer\\ \href{ftp://ftp.dante.de/tex-archive/macros/latex/contrib/beamer/doc/beameruserguide.pdf}{dante.de/tex-archive/macros/latex/contrib/beamer/doc/beameruserguide.pdf}
		\item Aktuellste Version dieses Paketes/Themes auf \href{http://github.com}{github.com}:\\
		      \href{https://github.com/Patschke/RPTU-Design/archive/master.zip}{https://github.com/Patschke/RPTU-Design/archive/master.zip}
	\end{itemize}
\end{frame}
\end{document}